% Term project, 2011 Jan Pokorny <xpokor04@stud.fit.vutbr.cz>

\setDocInfo{
    organization.cs       = Vysoké učení technické v~Brně,
    organization.en       = Brno University of Technology,
    organization.logo     = fig/req/vut-zp2,
    organization.logo.w   = 3.3,
    %
    subsidiary.cs         = Fakulta informačních technologií,
    subsidiary.en         = Faculty of Information Technology,
    subsidiary.logo       = fig/req/fit-zp2,
    subsidiary.logo.w     = 3.3,
    %
    department            = UITS,  % UPSY/UIFS/UITS/UPGM/...
    %
    project               = SP,    % BT/SP/MT/PT
    year                  = 2011,
    date                  = 11.\,ledna 2011,
    location              = Brno,
    %
    title.cs              = Symbolická exekuce C programů založená na Sparse,
    title.en              = Symbolic Execution of C Programs Based on Sparse,
    %
    author                = Jan Pokorný,
    author.title.p        = Bc.,
    %author.title.a        = Ph.D.,
    %author.email          = xpokor04@stud.fit.vutbr.cz,
    author.email          = pokorny\_jan@seznam.cz,
    %
    supervisor            = Kamil Dudka,
    supervisor.title.p    = Ing.,
    %supervisor.title.a    = Ph.D.,
    %
    assignment.img        = fig/req/assignment,
    %
    % Following can be set separately below
    %
    %abstract.cs           = abstract,
    %abstract.en           = abstract,
    %
    %keywords.cs           = keywords,
    %keywords.en           = keywords,
    %
    %copyright             = copyright,
    %declaration           = declaration,
    %acknowledgement       = acknowledgement,
}

%%%

\setAbstract[CS]{%
    Statická analýza je jednou z možností kontroly správnosti programů.
    Právě tímto směrem je orientována knihovna {\ttfamily code listener}
    poskytující rozhraní, jehož účelem je sjednotit výstup produkovaný
    dostupnými parsery kódu (zde představují přední části/front-endy
    celého systému) a nabídnout tak stabilní infrastrukturu pro navázání
    komponent provádějících další zpracování/analýzy (zadní části/back-endy
    systému).
    Cílem této práce je co nejvhodnějším způsobem na toto rozhraní přivést
    výstup produkovaný svobodným statickým analyzátorem {\ttfamily sparse},
    který pracuje nad zdrojovými kódy v jazyce C. Řešení si může
    vynutit případné změny v jeho kódu tohoto nástroje.
    Počítáno je i s eventuální potřebou provést zásahy do kódu
    tohoto nástroje.
    
    Finálním krokem pak bude porovnání s alternativním front-endem
    postaveným jako zásuvný modul pro překladač GCC.
}
\setAbstract[EN]{%
    Static analysis is one of the methods of programs checking.
    This approach is the main direction of {\ttfamily code listener}
    library that offers an API with endeavor to unify output produced
    of available code parsers (representing front-ends of the overall system)
    and hence to offer stable infrastructure for components doing further
    processing/analysis (system back-ends).
    The aim of this work is to brings output produced by the open source
    static analyzer {\ttfamily sparse} to this API, which may in turn
    also require changes in the code of this tool. Being capable of C
    code processing, {\ttfamily sparse} primarily serves for Linux kernel
    checking.

    The final step lies in the comparison of similar front-end
    already implemented as a GCC compiler plugin.
}

%%%

\setKeywords[CS]{%
    statická analýza, sparse, rozhraní code listener, symbolická exekuce
}
\setKeywords[EN]{%
    static analysis, sparse, code listener API, symbolic execution
}

%%%

\setCopyright{%
    Tato práce vznikla jako školní dílo na Vysokém učení technickém v Brně,
    Fakultě informačních technologií. Práce je chráněna autorským zákonem
    a její užití bez udělení oprávnění autorem je nezákonné, s výjimkou
    zákonem definovaných případů.
}

%%%

\setDeclaration{%
    Prohlašuji, že jsem tuto práci vypracoval samostatně pod vedením pana
    Ing.\,Kamila~Dudky.
    % Další informace mi poskytli...
    Uvedl jsem všechny literární prameny a publikace, ze kterých jsem čerpal.
}

%%%

%\setAcknowledgement{%
%    V této sekci je možno uvést poděkování vedoucímu práce a těm, kteří poskytli odbornou pomoc
%    (externí zadavatel, konzultant, apod.).
%}
