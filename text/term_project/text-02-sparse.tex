% Term project, 2011 Jan Pokorny <xpokor04@stud.fit.vutbr.cz>
%==============================================================================
\chapter{Nástroj \texttt{sparse}}
\label{chap:sparse}
%==============================================================================

Nástroj \texttt{sparse} je syntaktickým analyzátorem kódu v jazyce C
a jeho primární použití spočívá v prevenci programátorských chyb v jádře
operačního systému (OS) Linux. Další text bude brát jako referenční jeho
aktuální verzi \texttt{0.4.3} \cite{web:kernel.org:SparseDist}
(oznámení viz \cite{web:Gmane:Sparse043}).

Zde musím podotknout, že přímých zdrojů informací o \texttt{sparse} je poskrovnu,
takže nebude-li explicitně označeno, vycházím z vlastních zkušeností a poznatků
získaných ze zdrojových souborů v distribuci \texttt{sparse}, případně z informací
v bakalářské práci \cite{web:FITVUTBR:MartinNagy:StaticAnalysis}
a zdroje \cite{web:Wiki:Sparse}.

V úvodní části je \texttt{sparse} představen v obecnějším pojetí resp. pojetí
\uv{uživatele nástroje}, zatímco následující části pak představují pohled
\uv{uživatele stejnojmenné knihovny}, na níž je pak tento koncový nástroj postaven.


\section{Představení \texttt{sparse}}
\label{sec:jako-nastroj}

Vývoj \texttt{sparse}, z jehož názvu lze odvodit slovní spojení
\uv{semantic parse} neboli \emph{sémantický rozbor}, byl zahájen
Linusem Torvaldsem v roce 2003. Jedná se o statický analyzátor
kódu psaného v jazyce C zaměřený především na vyhledávání chyb
v linuxovém jádře (právě Torvalds stojí za zrodem Linuxu).

Jako takový se \texttt{sparse} skládá z obecnější knihovny schopné vstupní program
v podobě zdrojových souborů řízeně rozkládat na jeho syntaktické elementy
při zachování určitých sémantických informací o nich, a dále hlavního programu
\texttt{sparse}, jež tuto knihovnu používá jako svůj \emph{front-end}
(tj. prostředek pro vyzískávání vstupu, nad nímž manipulující program,
v této terminologii \emph{back-end}, teprve operuje).


\subsection{Spolupráce hlavního programu a knihovny \texttt{sparse}}

Po svém spuštění předá program \texttt{sparse} jednoduše řízení knihovně, a to
včetně argumentů svého spuštění. Ty určují jednak které soubory se mají knihovnou
předzpracovat, ale také jakým se způsobem se má toto zpracování provádět, tedy
především o kterém typu chyb určitě informovat a které naopak potlačit
(vztaženo k výchozímu nastavení knihovny). Za zmínku
stojí, že nástroj \texttt{sparse} potažmo příslušná knihovna byly navrženy pro
kompatibilitu s běžnými přepínači překladače GCC a dá se předpokládat,
že cílem byla mj. také snadná integrace kontrol do překladových systému jako
např. těch založených na programu \texttt{make}. To ostatně názorně ilustruje
právě případ linuxového jádra \cite{web:MjmWired:SparseTxt} nebo existence
skriptu \texttt{cggc} \cite{manual:cgcc}, který ze svých argumentů spuštění odebere ty,
které přísluší \texttt{sparse}, spustí nejprve běžný překladač a posléze
program \texttt{sparse} na tytéž zdrojové soubory.

Takto \texttt{sparse} provede inicializaci knihovny, která tímto také již vede
v patrnost makra, definice, případně dovezené další hlavičkové soubory (ty jsou
rovněž ihned zpracovány) a další konstrukce obsažené v souborech zadaných pomocí
argumentů uvozených přepínačem \texttt{-include}, jež budou spolu s předdefinovanými
symboly využity při zpracování skutečných vstupních souborů.
To probíhá v režii programu \texttt{sparse} tak, jak je bývá u back-endů
této knihovny zvykem%
%
\footnote{odráží se zde návrh knihovny}%
%
, tj. nechá knihovnu předzpracovat jeden soubor,
následně se s jejím využitím tento abstraktní syntaktický strom dále \uv{rozpracuje}
do takové míry, jak je třeba, a získané informace nějak využije.
Toto pak opakuje postupně pro všechny zbývající vstupní soubory.
V případě \texttt{sparse} spočívá toto finální zpracování v provedení některých dalších,
poměrně abstraktních kontrol, jako je např. hlídání správného použití
synchronizačních primitiv či prevence mísení uživatelského
o systémového adresového prostoru%
%
\footnote{projev orientace na komunitu vývojářů okolo linuxového
kernelu}%
%
.

%\begin{quote}
\noindent
\emph{Krátká odbočka}: Jelikož jazyk C není \textit{apriori} uzpůsoben, aby nesl nějakou
netriviální informaci o správném použití konstrukcí v něm definovaných%
%
\footnote{komentář v běžném jazyce analyzátorům typu \texttt{sparse} pochopitelně
nepomůže}%
%
, třeba právě v případě \emph{zámků}, zavádí se pro účely zpřístupnění
takovýchto informací pro \texttt{sparse} tzv.\,\emph{anotace kódu}. Více o nich
podá \cite{web:FITVUTBR:MartinNagy:StaticAnalysis}.
%\end{quote}

Přestože část kontrol zprostředkuje samotný \texttt{sparse} na základě
analyzovaných částí abstraktního syntaktického stromu, většina kontrol je na bedrech
knihovny, které jsou, jak bylo uvedeno, konfigurovatelné skrze předané
parametry. Týká se to všech vyhledávaných chyb lexikálních (např. neočekávaná
\emph{escape-sekvence} a syntaktických (např. neočekávaný výskyt
určitých \emph{tokenů} za sebou, např. dvou operátorů), a valné většiny
chyb sémantických (např. konstantní přístup za hranici pole).
Pokud v kódu nalezne potenciálně chybnou konstrukci a varování pro tento
druh chyb přitom není potlačeno, je varovná zpráva vypsána přímo
na standardní chybový výstup.


\subsection{Shrnutí hlavních myšlenek \texttt{sparse}}

Výhoda tohoto vyčlenění knihovny a \uv{obálky} nad ní spočívá v možnosti
používat  knihovnu při dodržení licenčních podmínek
(\uv{\emph{Open Software License v1.1}})
k vlastním účelům, kdy je potřeba převést zdrojové
texty v jazyce C do odpovídající a dále dobře zpracovatelné
interpretace včetně dostatku přímých i odvozených
podrobností o kódu. Tím si lze ulehčit práci při psaní nadstavbového
nástroje pro formální analýzu (což je i částečný cíl této práce) stejně
dobře jako při konstrukci vlastního kompilátoru. Není divu, že se
v distribuci \texttt{sparse} nachází další programy\ndash\emph{back-endy},
jež demonstrují, k čemu je možné knihovnu využít.
%Stručnou ukázku použití každé včetně samotného \texttt{sparse} uvádí
%příloha \ref{app:sparse-backends}.

Jak z výše uvedeného částečně vyplývá, detailní rozbor kódu
na popud \emph{back-endu} neproběhne ihned s okamžitým vyjádřením
všech podrobností, z počátku je k dispozici informace pouze o symbolech
nejvyšší úrovně (tj. včetně funkcí), na základě níž pak může
\emph{back-end} prozkoumávat abstraktní syntaktický strom do hloubky a získávat
detaily o konkrétních uzlech k tomu určenými knihovními funkcemi.
Část informací z tohoto stromu je vyhodnocena teprve až na žádost
o její získání učiněnou skrze některou z těchto funkcí
(tzn. princip \emph{líného vyhodnocení}). Toto spolu s dalšími principy
(např. \uv{soubor je nedělitelná jednotka zpracování}) podporuje
ideály, které Torvalds v počátku projektu stanovil, tj. aby
knihovna byla co nejprostší, snadno
použitelná, nezávislá na dalších projektech, ale přitom zbytečně
nenabobtnalá (viz také \texttt{README} v distribuci \texttt{sparse}).


\section{Vnitřní činnost \texttt{sparse}}

Pozornost nyní přesuneme od obecnějšího pojednání o \texttt{sparse}
k částem věnovaným \uv{vnitřku}, tedy jakým způsobem knihovna pracuje,
jaké datové struktury k tomu využívá a podobně. Začneme od nejvyšší\ndash{}přehledové
úrovně a později se budeme věnovat detailům.

Faktem je, že pro neseznámeného je alespoň částečné pochopení vnitřního
chodu knihovny i přes dostupnost zdrojových kódů poměrně obtížně.
Je to důsledkem jednak citelného nedostatku dostatečně
ozřejmňujících komentářů v kódu a jednak neexistencí nějaké
dokumentace, natož podrobnější.

Pro lepší pochopení souvislostí
zejména mezi funkcemi a dále pro rychlou orientaci v kódu typu
\uv{mám název funkce a chci vidět, v jakém kontextu se ještě
používá} jsem se rozhodl ručně sepsat hierarchickou posloupnost
volání funkcí, jaká odpovídá běžné činnosti programu \texttt{sparse}.
Výsledek uvádím také v příloze \ref{app:call-hieararchy}%
%
\footnote{v samostatném textovém souboru se používá podstatně lépe!}%
%
.
Posloužil mi pro seznamování se s kódem a také posloužil
pro vizualizaci toho, která funkce má větší význam a které jsou
naopak čistě pomocné.

\section{Pohled na činnost knihovny shora}

Pro reprezentaci činnost knihovny na nejvyšší úrovni
při jejím typickém použití včetně návaznosti na okolní
prostředí jsem zvolil nákres, který je na obr.\,\ref{fig:chod} na str.\pageref{fig:chod}.

\hspace*{\fill}\\[-\baselineskip]
\begin{figure}[!h]
    \begin{center}
        \includegraphics[width=1\textwidth,keepaspectratio]{fig/schema_chodu}
        \label{fig:chod}
        \caption{Schéma reprezentující činnost knihovny \texttt{sparse} na nejvyšší úrovni
                 při typickém použití}
    \end{center}
\end{figure}

Jsou v něm dobře patrné fáze, kterými kód během zpracování prochází:
\begin{enumerate}
    \item\label{lex} lexikální analýza, výsledkem jsou tokeny (viz funkce \texttt{tokenize()})
    \item průchod preprocesoru, výsledkem jsou upravené tokeny (viz funkce \texttt{preprocess()})
    \item syntaktická analýza, výsledkem je seznam symbolů,
          které ve stromech, které reprezentují, obsahují různé
          příkazy a výrazy (v terminologii jazyka C)
    \item\label{sym} symboly doplněné o informaci o typu (viz funkce \texttt{evaluate\_symbol\_list()},
          týká se pouze vstupních souborů)
    \item zjednodušené symboly, což se týká např. zjednodušení aritmetických výrazů
          a porovnání v případech, kde to lze (viz funkce \texttt{expand\_symbol()})
    \item funkční symboly převedené na vstupní body (\emph{entry points})
          a jejich těla transformovaná na graf základních bloků (\emph{basic blocks}),
          %
          \footnote{lineárně uspořádaná posloupnost příkazů/instrukcí bez \uv{odskoků} jinam}
          %
          tj. graf toku řízení (\emph{control flow graph}) (viz funkce \texttt{linearize\_symbol()}
          potažmo funkce \texttt{linearize\_fn()})
\end{enumerate}

Rovněž lze dobře vidět, že fáze \ref{lex} až \ref{sym} probíhají z pohledu
uživatele knihovny atomicky, čili snažit se ovlivnit tuto část zpracování
jinak než pomocí argumentů předaných funkci \texttt{sparse\_initialize()}
by patrně přineslo mnohé komplikace.

Dále se zaměříme na výstup poslední uvedené fáze zpracování, a to na linearizovaný
kód a jeho reprezentaci.

\section{Linearizovaný kód a jeho reprezentace}

Vrchní úroveň linearizovaného kódu představuje struktura \texttt{entrypoint}
definovaná, tak jako ostatní struktury podílející se na této
reprezentaci kódu, v souboru \texttt{linearize.h}. Výčet jejích položek
zachycuje tab.\,\ref{tab:entrypoint}.

\begin{table}[!h]
    \begin{center}
        \begin{tabular}{|l|l|}
            \hline
            \textsl{název položky} & \textsl{význam} \\
            \hline
        	    \texttt{struct symbol *name} & odpovídající symbol \\
            	\texttt{struct symbol\_list *syms} & seznam využitých symbolů \\
            	\texttt{struct pseudo\_list *accesses} & seznam využitých \uv{pseudoproměnných} \\
             	\texttt{struct basic\_block\_list *bbs} & seznam všech bloků v grafu toku řízení \\
                \texttt{struct basic\_block *active} & příslušný \emph{basic blok} \\
                \texttt{struct instruction *entry} & instrukce typu \texttt{OP\_ENTRY} \\
            \hline
        \end{tabular}
        \caption{položky pro \texttt{struct entrypoint}}
        \label{tab:entrypoint}
    \end{center}
\end{table}

Jednu z položek struktury \texttt{entrypoint} tvoří struktura \texttt{basic\_block},
jejíž položky uvádí tab.\,\ref{tab:basicblock}.

\begin{table}[!h]
    \begin{center}
        \begin{tabular}{|l|l|}
            \hline
            \textsl{název položky} & \textsl{význam} \\
            \hline
            	\texttt{struct position pos} & příslušný soubor a pozice v něm \\
                \texttt{unsigned long generation} & položka použita při eliminaci mrtvého kódu \\
                \texttt{int context} & \\
                \texttt{struct entrypoint *ep} & příslušný vstupní bod \\
                \texttt{struct basic\_block\_list *parents} & \emph{basic bloky}, které směřují do tohoto\\
                \texttt{struct basic\_block\_list *children} & \emph{basic bloky}, kam lze přejít z tohoto \\
	            \texttt{struct instruction\_list *insns} & příslušný seznam instrukcí \\
	            \texttt{struct pseudo\_list *needs} & na kterých \emph{pseudoproměnných} tento blok závisí \\
                \texttt{struct pseudo\_list *defines} & které \emph{pseudoproměnné} tento blok definuje\\
            \hline
        \end{tabular}
        \caption{položky pro \texttt{struct basic\_block}}
        \label{tab:basicblock}
    \end{center}
\end{table}

Struktura \texttt{basic\_block} se odkazuje mj. na strukturu \texttt{instruction},
jejíž položky uvádí tab.\,\ref{tab:instruction}.

\begin{table}[!h]
    \begin{center}
        \begin{tabular}{|l|l|}
            \hline
            \textsl{název položky} & \textsl{význam} \\
            \hline
	            \texttt{unsigned opcode:8} & kód instrukce určující, co provádí\\
            	\texttt{unsigned size:24} & velikost operandů instrukce \\
                \texttt{struct basic\_block *bb} & \\
                \texttt{struct position pos} & příslušný \emph{basic blok}\\
                \texttt{struct symbol *type} & typ výsledné \emph{pseudoproměnné}\\
            \hline
        \end{tabular}
        \caption{položky pro \texttt{struct instruction}}
        \label{tab:instruction}
    \end{center}
\end{table}
