% Term project, 2011 Jan Pokorny <xpokor04@stud.fit.vutbr.cz>
%==============================================================================
\chapter{Závěr}
\label{chap:zaver}
%==============================================================================

Byly zde ukázány některé podklady pro další pokračování na diplomové práci.
Bohužel se nepodařilo zcela dodržet předpokládanou hloubku zpracování
bodů zadání, což je dáno zejména tím, že se teprve postupně ukázalo, že
\texttt{sparse} je komplikovanější, než se zpočátku jevilo, a tedy
nelze jím procházet jedním směrem bez pochopení celého kontextu.
Prvním krokem bude hlubší vstřebání detailní činnosti knihovny \texttt{sparse}
následované prvním prototypem požadovaného adaptéru. Půjde vlastně o provázanou
činnost, neboť pro zdokonalení tohoto prototypu bude potřeba prozkoumat některá
dosud skrytá zákoutí knihovny a naopak experimentování s prototypem přispěje
k lepšímu pochopení jejího fungování.

Považuji za vhodné část práce vyhradit právě popisu vnitřního chodu knihovny,
protože až na jednu výjimku se mi nepodařilo žádný takový zdroj nalézt
a komentářů ve zdrojových souborech, které by umožnili rychle se v nich
zorientovat, se bohužel příliš nedostává.

V případě, že se ukáže, že implementace není zase tak obtížná a tudíž
bude odladěná implementace hotova brzo, nabízí se po dohodě s vedoucím
práce možnost rozšíření v podobě implementace komponenty na opačné straně
rozhraní \texttt{code listener}. Může se však stát, že pro úspěšné dokončení
primárního cíle práce bude nutné provést koordinovaně s komunitou kolem
\texttt{sparse} určité úpravy tohoto nástroje \ndash\ v tomto případě
počítám, že řešení bude časově náročnější. V každém případě bude nutné,
abych na vytvořený nástroji demonstroval schopnosti odhalování chybné
manipulace s pamětí a provedl srovnání s existující alternativou
v podobě zásuvného modulu pro překladač \texttt{GCC}.
