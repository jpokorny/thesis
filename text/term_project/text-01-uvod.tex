% Term project, 2011 Jan Pokorny <xpokor04@stud.fit.vutbr.cz>
%==============================================================================
\chapter*{Úvod}
\addcontentsline{toc}{chapter}{Úvod}
\label{chap:uvod}
%==============================================================================

Tato práce představuje jakýsi vstup do problematiky, jíž se pak dále
zabývá navazující diplomová práce týká. Konkrétně je zde dán prostor
pro zpracování prvních dvou bodů jejího zadání, které znějí (viz také kopie
celého zadání vložená za titulní stranu):

\begin{itemize}
    \item Seznamte se s nástrojem \texttt{sparse} pro statickou analýzu kódu v jazyce C
          a rozhraním \texttt{code listener} pro analýzu zdrojového kódu.
    \item Prostudujte interní reprezentaci linearizovaného kódu v nástroji \texttt{sparse}.
          Zaměřte se zejména na reprezentaci instrukcí pro práci s pamětí.
\end{itemize}

Vzhledem ke dvěma zmíněným softwarovým celkům je také obsah této práce rozdělen do
dvou částí \ndash\ v kapitole \ref{chap:sparse} je pojednáno o nástroji \texttt{sparse}
a kapitola \ref{chap:code-listener} se pak věnuje rozhraní \texttt{code listener}.
V závěru jsou pak uvedeny zejména vyhlídky na další pokračování projektu
pro účely diplomové práce.
